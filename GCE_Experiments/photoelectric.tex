\documentclass[11pt,a4,epsf]{jarticle}
\usepackage{latexsym}
\usepackage{mathrsfs}
%\usepackage{url}
%\usepackage{lscape}
\usepackage{graphics}
\usepackage{theorem}

%\newtheorem{definition}{Definition}

%% for apple LaserWriter Series %%
%% 
\setlength{\topmargin}{-0.5in}
\setlength{\textwidth}{5.6in}
\setlength{\textheight}{8.8in}
\setlength{\oddsidemargin}{0.35in}
\setlength{\evensidemargin}{0in}

\usepackage{theorem}
\renewcommand{\baselinestretch}{1.2}
\setlength{\parskip}{0.25ex}
\renewcommand{\arraystretch}{0.85}

\title{光電効果とプランク常数の決定}
\author{おれおれ.}
\date{}

\begin{document}

\maketitle

\section{光電効果}

多くの金属で,その表面に光(電磁波)をあてると電子が飛び出す現象がおきる.
これを光電効果 photoelectric effect という.

マクスウェルは方程式から電磁波の存在を予測していたが,
ヘルツ H. Hertz はこれを確かめる試みの中でこの現象を 1887 年に発見した.
しかしその現象は,光を波動として説明するマクスウェルの古典的電磁気学ではなく,
プランク M. Panck のエネルギー量子の考え方を導入したアインシュタイン A. Einstein の光量子仮説(1905年)によって説明された.
この理論は,1914 年にミリカン R. Millikan の詳しい実験によって裏付けられた.

\section{原理}

光電管は,光をあてることのできる金属板と集電極が,電極として真空中にはなれておかれた構造になっている.
金属板に光をあてると,金属の表面から光電子が放出され,これらが電極に到達すれば,
電極間にながれる電流として検出される.

ここで,金属版を陰極として電圧を印加すれば,放出された電子は陽極である集電極にあつまる.
逆に金属版を陽極として電圧を印加した場合,
放出された電子の持つ電極方向の運動エネルギーが,
電位差に逆らって電極に到達することによって失うエネルギー以上であるとき,はじめて集電極に到達し,電流が検知されることになる.
すなわち,集電極方向の電子の運動エネルギー $K = \frac{1}{2}m_{\rm e} v^2$ が $K \geq {\rm e}|V|$ であるとき,電流が流れる.ここで $m_{\rm e}$ は電子の質量,${\rm e}$ は電子のもつ電荷の大きさである.
金属板にあてる光を単色光とし,金属板を陽極として印加する電圧を高くしていくと,ある電圧で電流が 0 となる.
この電圧を,制止電圧という.
これは,単色光によって金属板から放出される電子の運動エネルギーの最大値 $K_{\max}$ と制止電圧 $V_0$ について $K_{\max} = {\rm e}|V_0|$ が成り立つ状態にあるということである.

プランクは,黒体輻射の現象 --- 
熱せられた物体が放射する光の強さと振動数の関係 --- を説明するため,
光として放出されるエネルギーがとびとびの値をとる,エネルギー量子仮説を提唱した.
これによれば,振動数 $\nu$ の光により放出されるエネルギーの値は,$h \nu$ の整数倍 $n h \nu$ になる.
ここで $n$ は正の整数であり,また振動数 $\nu$ である光の強度といえる.
光電効果は輻射と逆の現象,つまり光が金属板にあたり,そのエネルギーが吸収されて電子が放出される現象である.
その電子の放出は,金属の面に光があたってから $10^{-9}$ 秒(1ナノ秒)より短い時間内におこり,照射から射出までに時間のおくれはない\cite{Aya-}.
つまりエネルギーの蓄積は行われていないと考えられるので,
金属板から放出される電子の運動エネルギーの最大値 $K_{\max}$ は,光のエネルギー量子の大きさに等しいと考えられる.
振動数 $\nu$ の光を照射したときの制止電圧が $V_0$ ならば,
\[
K_{\max} = \mbox{e} V_0 = h \nu + c
\]
が成り立つことになる.
ここで,$c$ は電子が金属の束縛を逃れるために必要なエネルギーや,制止電圧のずれなどを考慮した定数(正または負)とする.

つまり,同じ光電管(の金属板)について,同じ回路で,照射する光の振動数 $\nu$ を変化させて制止電圧を測定すれば,振動数と制止電圧の関係から比例定数であるプランク定数 $h$ を求めることができる.

\section{実験}

まず,全体を通して使用する光電効果実験装置について,注意事項,および基本的な手順を付録 Appendix \ref{apdx:aparatus} で確認する.
そのうえで,実験 \ref{a} を行って実際に手順を確認する.
これら基本的な手順と測定方法を確認した後,実験 \ref{b} および \ref{c} を行う.


\subsection{光電流現象の確認,電極間印加電圧と光電流}\label{a}

光電流が流れることを確認せよ.
光の照射強度は,LED 強度つまみを変化させる,もしくは LED モジュールへの電力供給プラグを抜き差しすることで ON/OFF させる.
また,電圧調整つまみで制止電圧を変化させ,光電流が正または負で流れる値にする.

現象を確認し,また光電流が流れる大まかな値を把握する.

\subsection{光の強度と光電流}\label{b}

\subsection{光の波長と阻止電圧}\label{c}

\appendix
\section{}\label{apdx:aparatus}

\end{document}
