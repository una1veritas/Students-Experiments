%#! pdflatex main


In this section, we will give a reduction that shows $\MinConDFAprefSet$ is $\classNP$-hard in Theorem~\ref{theorem:intractability}.
It is also a main part of the proof of non-approximability in Theorem~\ref{theorem:nonapproximability}.
At first, we give a transformation procedure employed in our reduction from the graph coloring problem.

Let $\Sigma = \{\pathz, \patho\}$ and $\Gamma = \{\posi, \nega,\wild\}$.
For notational convention, we define a {\em labeled string} by 
a pair $(s,r)$ of strings $s \in \Sigma^*$ and $r \in \Gamma^*$ of the same
length.
It corresponds to the set $\textit{Pos} = \{ s[0\!:\!i] \mid r[i] = \posi\}$ 
of positive examples and the set 
$\textit{Neg} = \{ s[0\!:\!i] \mid r[i] = \nega\}$ of negative examples in an
obvious way.
The {\em concatenation} of two labeled strings $u_1 = (s_1,r_1)$ and
$u_2 = (s_2,r_2)$ is
naturally defined by $u_1\!\cdot\!u_2 = (s_1\!\cdot\! s_2,r_1\!\cdot\! r_2)$.

Given a graph $G$ of an instance of \MinGC, 
we translate it into a sample represented by a labeled string $S_G$ of an instance of $\MinConDFAprefSet$.
First, we describe how the adjacency matrix of $G$ will be encoded in labeled strings. 

For each vertex $i\in V$ of a graph $G=(V,E)$, we associate a labeled
string $g_i = (\pathz^{2\sizeV}, \tilde{\ell_i})$, where $ \tilde{\ell_i} = \ell_i[0]\cdot \ell_i[1]\cdot\ell_i[2]\cdots
\ell_i[n-1]$ and, for each $k \in V$, 
\[
% \tilde{\ell_i} = \ell_i[0]\cdot \ell_i[1]\cdot\ell_i[2]\cdots
% \ell_i[n-1]\, \mbox{, \quad and \quad}
%\]
%%\[
\ell_i[k] = 
\begin{cases}
\posi\nega & \mbox{ if $k=i$},\\
\nega\nega & \mbox{ if $(i, k) \in E$},\\
\wild\nega & \mbox{ otherwise. }
\end{cases}
\]
%
We call this $g_i$ for a vertex $i$ an {\em adjacency coding of $i$}. 

Notice that, for each $i \in V$, the label $\posi$ exists only at $\ell_i[i]$. 
Also, the substring $\nega\nega$ occurs at columns where adjacent vertices exist. 

\begin{lemma}
%	For any $i,j \in V$, 
%	\begin{enumerate}
%		\renewcommand{\labelenumi}{(\arabic{enumi})}
%		\item $(i,j) \in E$ if and only if there exists $k \in V$ such that $\ell_i[k]=\nega\nega$ and $\ell_j[k]=\posi\nega$.
%		\item $g_i$ and $g_j$ can be merged if and only if $(i,j) \notin E$.
%	\end{enumerate}
Labeled strings $\tilde{\ell_i}$ and $\tilde{\ell_j}$ of adjacency codings $g_i, g_j$ for $i, j \in V$ have both $\posi$ and $\nega$ at the same position if and only if $(i,j) \in E$. 
\label{lem:joint2}
\end{lemma}
\begin{proof}
\newcommand{\set}[1]{\{#1\}}
% \begin{enumerate}
	% \renewcommand{\labelenumi}{(\arabic{enumi})}
%	(1) % \item
%	If $(i,j) \in E$, we have $\ell_i[j]=\ell_j[i]=\nega\nega$.
%	Otherwise, $\ell_j[k]=\posi\nega$ holds only at $k=j$.
%	For the opposite, $\ell_i[j]=\ell_j[i]=\wild\nega$ since $(i,j) \notin E$.
%
%	(2) % \item
%	By the proof of (1), 
If $(i,j) \notin E$, $g_i$ and $g_j$ have no position $k$ such that $\ell_i[k]=\nega\nega$ and $\ell_j[k]=\posi\nega$, nor $\ell_i[k]=\posi\nega$ and $\ell_j[k]=\nega\nega$.
%	So these two adjacency codings are consistent, and can be merged into one coding. \qed
	% Then they can be merged into one coding $g_{\set{i,j}}=(a^{2\sizeV},\tilde{\ell}_{\set{i,j}})$ , where $\tilde{\ell}_{\set{i,j}} = \ell_{\set{i,j}}[0]\cdot \ell_{\set{i,j}}[1]\cdot\ell_{\set{i,j}}[2]\cdots \ell_{\set{i,j}}[\sizeV-1] $ and for each $k \in V$, 
	% \begin{eqnarray*}
	% 	\ell_{\set{i,j}}[k] = 
	% 	\begin{cases}
	% 	\posi\nega & \mbox{ if $k \in \set{i,j}$},\\
	% 	\nega\nega & \mbox{ if $\ell_h[k] = \nega\nega$ for some $h \in \set{i,j}$},\\
	% 	\wild\nega & \mbox{ otherwise. }
	% 	\end{cases}
	% \end{eqnarray*} \qed
% \end{enumerate}
\qed
\end{proof}
%
Therefore, if $g_i$ and $g_j$ are in a labeled string as parts and $(i,j) \not\in E$, then the prefixes ending within those parts can be correctly classified by one state transition sequence of length $2n$. 
In the following, we say that such labeled strings are {\em consistent}, and able to be {\em merged}. 

By Lemma~\ref{lem:joint2}, a subset $V' \subseteq V$ is an independent set of graph $G$ if and only if $g_i$ and $g_j$ can be merged  for any $i, j \in V'$.
A {\em $k$-partition} $V_0, V_1, \ldots, V_{k-1}$ of $V$ is a sequence of mutually disjoint$k$ subsets of $V$ whose union equals $V$.
We identify a $k$-partition of $V$ with a mapping $f:V \rightarrow \{ 0,1,\cdots,k-1 \}$.
$f$ is a $k$-coloring of $G$ if and only if, for any $h \in \{ 0,1,\cdots,k-1 \}$ and $i,j \in V_h$, $g_i$ and $g_j$ can be merged.
We denote the {\em merged adjacency coding} for a subset $V'$ by $g_{V'}=(\pathz^{2\sizeV}, \tilde{\ell}_{V'})$, where $\tilde{\ell}_{V'} = \ell_{V'}[0]\cdot \ell_{V'}[1]\cdots \ell_{V'}[\sizeV-1] $ with for each $k \in V$,
\[
	\ell_{V'}[k] = 
	\begin{cases}
	\posi\nega & \mbox{ if $k \in V'$},\\
	\nega\nega & \mbox{ if $\ell_j = \nega\nega$ for some $j \in V'$},\\
	\wild\nega & \mbox{ otherwise. }
	\end{cases}
\]

For preparation,
we denote by $\ident{i}$ the labeled string $(\Bin_i, \nega^t)$ and call it {\em binary coding},
where $t=\ceilbrace{\log{\sizeV}}$ and $\Bin_i$ is the binary representation of $i$ by using characters in $\Sigma$ of length $t$.
For instance, if $\sizeV=6$, we have 
$\ident{0} = (\pathz\pathz\pathz, \nega\nega\nega)$, 
$\ident{1} = (\pathz\pathz\patho, \nega\nega\nega)$, 
$\ident{2} = (\pathz\patho\pathz, \nega\nega\nega)$, 
$\ldots$ 
$\ident{6} = (\patho\patho\pathz, \nega\nega\nega)$.
We now define the labeled string $S_G$ by
%	we convert graph $G$ and positive integer $k$ to
%\[
$S_G = s_0 \cdot s_1 \cdot s_2 \cdots s_{\sizeV-1}$,
%\]
%$d =(\patho, \nega)$ plays a role of delimiter, and 
%each $s_i \ (i=0,1,2,\cdots,n-1)$ is defined by
%\[
%\]
where $s_i = \ident{i}\cdot\head\cdot g_i \cdot \tail$ with $\head = (\pathz^{\sizeV^2}, \nega^{\sizeV^2})$ and $\tail =
(\pathz\pathz\patho,\posi\posi\nega)$ for each $i \in V$.
(See Fig.~\ref{fig:exampleSg2}.)
%	$\Sigma_L$-string $S_G$ and positive integer $p$.
The length is
\[
 |S_G| \, = \, \sizeV( \ceilbrace{\log{\sizeV}} + \sizeV^2 + 2\sizeV + 3) 
  \, = \, \sizeV^3 + 2\sizeV^2 + \sizeV\ceilbrace{\log{\sizeV}} + 3\sizeV.
\]
%
%Fig.\ref{fig:exampleSg2} shows an example.
%
%
\begin{figure}[tbp]
%\begin{minipage}{3cm}
%\fbox{
\begin{picture}(60,85)(20,5)
%\put(0,0){(0,0)}
%\put(100,0){(100,0)}
%\put(0,100){(0,100)}
%\put(70,70){(70,70)}
\put(50,80){\circle{12}}
\put(48,77){0}
\put(80,60){\circle{12}}
\put(78,57){1}
\put(80,30){\circle{12}}
\put(78,27){2}
\put(50,10){\circle{12}}
\put(48, 7){3}
\put(20,30){\circle{12}}
\put(18,27){4}
\put(20,60){\circle{12}}
\put(18,57){5}
%
%\linethickness{1mm}
\put(56,77){\line(3,-2){19}} % edge (0, 1)
\put(80,54){\line(0,-1){18}} % edge (1, 2)
\put(56,13){\line(3, 2){19}} % edge (3, 2)
\put(44,13){\line(-3, 2){19}} % edge (3, 4)
\put(20,54){\line(0,-1){18}} % edge (5, 4)
\put(44,77){\line(-3, -2){19}} % edge (0, 5)
\put(75,57){\line(-2,-1){49}} % edge (1, 4)
\put(26,30){\line(1,0){48}} % edge (4, 2)
\end{picture}
%}
%\end{minipage}
%
{ \tabcolsep=4pt
\renewcommand{\arraystretch}{0.6}
\newcommand{\zeroooo}{\mbox{\tt 00000}\ldots\mbox{\tt 000}}
\newcommand{\aaaaaaa}{\mbox{\tt aaaaa}\ldots\mbox{\tt aaa}}
\def\gray{}
{\tt 
\begin{tabular}{>{\columncolor[gray]{0.9}}c>{\columncolor[gray]{0.9}}cc>{\columncolor[gray]{0.9}}cc>{\columncolor[gray]{0.9}}cc>{\columncolor[gray]{0.9}}cc>{\columncolor[gray]{0.9}}cc>{\columncolor[gray]{0.9}}cc>{\columncolor[gray]{0.9}}c>{\columncolor[gray]{0.9}}c>{\columncolor[gray]{0.9}}cc}
\rowcolor{white}$\ident{i}      $ & \head & $0$ && $1$ && $2$ && $3$ && $4$ && $5$ && \tail \\ \hline
\rowcolor[gray]{.9}aaa & $\aaaaaaa$         & a & a & a & a & a & a & a & a & a & a & a & a & aab \\
000 & $\zeroooo$         & \cellcolor[gray]{0.9}1 & 0 & 0 & 0 & * & 0 & * & 0 & * & 0 & 0 & 0 & 110 \\
\rowcolor{white}\\[-2.5mm]
\rowcolor[gray]{.9}aab & $\aaaaaaa$         & a & a & a & a & a & a & a & a & a & a & a & a & aab \\
000 & $\zeroooo$         & 0 & 0 & \cellcolor[gray]{0.9}1 & 0 & 0 & 0 & * & 0 & 0 & 0 & * & 0 & 110 \\
\rowcolor{white}\\[-2.5mm]
\rowcolor[gray]{.9}aba & $\aaaaaaa$         & a & a & a & a & a & a & a & a & a & a & a & a & aab \\
000 & $\zeroooo$         & * & 0 & 0 & 0 & \cellcolor[gray]{0.9}1 & 0 & 0 & 0 & 0 & 0 & * & 0 & 110 \\
\rowcolor{white}\\[-2.5mm]
\rowcolor[gray]{.9}abb & $\aaaaaaa$         & a & a & a & a & a & a & a & a & a & a & a & a & aab \\
000 & $\zeroooo$         & * & 0 & * & 0 & 0 & 0 & \cellcolor[gray]{0.9}1 & 0 & 0 & 0 & * & 0 & 110 \\
\rowcolor{white}\\[-2.5mm]
\rowcolor[gray]{.9}baa & $\aaaaaaa$         & a & a & a & a & a & a & a & a & a & a & a & a & aab \\
000 & $\zeroooo$         & * & 0 & 0 & 0 & * & 0 & 0 & 0 & \cellcolor[gray]{0.9}1 & 0 & 0 & 0 & 110 \\
\rowcolor{white}\\[-2.5mm]
\rowcolor[gray]{.9}bab & $\aaaaaaa$         & a & a & a & a & a & a & a & a & a & a & a & a & aab \\
000 & $\underbrace{\zeroooo}$
                         & 0 & 0 & * & 0 & * & 0 & * & 0 & 0 & 0 & \cellcolor[gray]{0.9}1 & 0 & 110 \\
% \rowcolor{white}& \underbrace \\[-2.5mm]
\rowcolor{white} & $36$ 
\end{tabular} \vspace{-3mm}
}}
\caption{A labeled string $S_G$ translated from the graph $G=(V,E)$ on the left.
Remark that $S_G$ is a single labeled string, although it is written in 6 lines for easy understanding.
Shadowed symbols depend only on the number $|V|$ of the vertices, and
only unshadowed symbols reflect the connectivities of the edges in $E$.
% A gray part is fixed by vertex size $n$. A white part changes by links between vertices.}
}
\label{fig:exampleSg2}
\end{figure}
\renewcommand{\arraystretch}{1.0}

We now consider DFAs that are consistent with $S_G$.
Obviously, any DFA of the form as illustrated in Fig.~\ref{fig:minimizeDFA} (top) is consistent with
$S_G$; here, double circles are accepting states, black circles are
rejecting states, and single circles can be either accepting or
rejecting states.
By merging some states consistently, we can get a smaller DFA.
For instance, a DFA $\tilde{M}_G$ in Fig.~\ref{fig:minimizeDFA} (bottom)
is also consistent with $S_G$.
%Now suppose that a partitioning DFA $\tilde{M}_G$ is 
We show that $\tilde{M}_G$ has a unique structure which is easily transformed into coloring $\Phi$, and which is specified by {\em necks} and {\em streams} defined as follows.

% $\tilde{M}_G$ has accepting states that are the last symbols of $head$s.
The {\em necks} are the last symbols of $head$s in $\tilde{M}_G$.
They contribute to extract coloring from $\tilde{M}_G$.
% We call those states {\em necks}.
% Before looking into the proof of our main result, 
% we show any partitioning DFA has a similar structure specified by necks.
% or can be transformed in log-space from any DFA consistent with $S_G$ without increasing the number of states.
% 
% $\tilde{M}_G$ has acyclic chained states which is called {\em streams}.
A {\em stream} is an acyclic chained states linked only by $\pathz$-transitions.
It consists of head, adjacency coding and a part of tail.
 % and where transition is only $\pathz$-transition.
% In another word, any transition by $\patho$-transition instantly leads to rejection,
% although a state on a stream can be either an accepting state or a non-accepting state. 
Note that no stream can have any transition branching off with $\pathz$-transition because the automaton is deterministic.


\begin{property}\label{pro:head}
	Every neck appears after a stream with $\sizeV^2 -1$ accepting states handling the head parts. 
	All the streams flowing into the same neck can be merged into one stream, without increasing the number of states. 
\end{property}
\begin{proof}
	Any transitions run over by symbols of the head cannot make a closed cycle; 
	if there is a cyclic transition, it must merge all the following `$\pathz$'s including those in the adjacency coding and the tail; 
	then $M_G$ is inconsistent with $S_G$ because the tail part must (and also possibly the body part) have a rejecting state.
	Since all the streams down to one neck are equivalent --- the states at the same distance to the neck form a right-invariant group ---, all those streams can be merged into one. \qed
\end{proof}

\begin{property}\label{pro:acyclic}
	Any transition run over by an adjacency coding and the following tail (except a last few symbols) forms a stream partially.
\end{property}
\begin{proof}
	An adjacency coding part has accepting states on alternate states, but the following tail runs over any such cycle. 
	The tail has a unique accepting state sequence $(\pathz\pathz\patho, \posi\posi\nega)$, which never appears in adjacency coding.
	Thus a cycle run over by the adjacency coding and the tail makes a contradiction.

	Note that it is possible that the transitions for a few last symbols can be merged into some stream, and then leaves by $\patho$-transition. \qed
\end{proof}

\begin{property}\label{pro:tail}
	All the tails can be merged into one stream without increasing the number of states. 
	That merge can extend to upstream including a few states of adjacency codings at their end. 
\end{property}
\begin{proof}
	In order to minimize the binary representations, we must merge them like a binary tree.
	The root node of the tree is the initial state of the binary representations, i.e., the last states of the tails.
	If all tails are not merged, binary representations cannot form a binary tree.

	Streams down to the root node are equivalent because all of them are tails, so they are merged into one stream.
	Furthermore, if the last few states in some combination of adjacency coding flowing into tail is equivalent, they can be also merged into one stream respectively. \qed
	%
	% Tail has accepting state sequence which don't appear in binary coding, head nor adjacency coding.
	% So they are merged into one stream.
	% Note that the last accepting state of tail can be merged with another accepting state in adjacency coding or tail,
	%  and leaves from stream by $\patho$-transition.
	%
	% If the last few states in some combination of adjacency coding flowing into tail is equivalent, they can be also merged into one stream respectively. \qed
\end{proof}

%\snote{The `information states' of DFA regarding vertex-id parts and then distributing sequences into streams 
%may have various structure, not limited to dags. 
%This will not contradict with the discussions on streams given above. 
%The number of information states will be carefully discussed later. }
%
%Let $M_G$ be a DFA consistent with the labeled string $S_G$ and in a $k$-neck normal form. 
%Then $G$ is $k$-colorable, and the size of $M_G$ satisfies $LB(\sizeV,k) \leq |M_G| \leq UB(\sizeV,k)$, where
%}
\begin{lemma} \label{lemma:consistent DFA UB and LB, version 2}
% Let $G=(V,E)$ be any graph of $n$ nodes, which can be $k$-colored
% , and
% let $M_G$ be the smallest DFA that is consistent with the labeled string $S_G$.
Let $G=(V,E)$ be a graph with $\sizeV$ vertices, and $M_G$ a DFA that is consistent with $S_G$. 
Let $k$ be the number of states of $M_G$ to which any character at the last of heads visits. 
Then, (i) with a log-space computation, we can obtain a partitioning DFA $\tilde{M}_G$ whose size is no greater than $M_G$, and in which all characters in heads are treated by one of $k$ chains of $\sizeV^2$ transitions (Fig.~\ref{fig:minimizeDFA}), and (ii) the size $|\tilde{M}_G|$ satisfies the inequality $LB(\sizeV,k) \leq |\tilde{M}_G| \leq UB(\sizeV,k)$, where 
%Then the size of $M_G$ is bounded by $LB(\sizeV,k) \leq |M_G| \leq UB(\sizeV,k)$, where
\begin{eqnarray*}
LB(\sizeV,k) %	&=& ( \ceilbrace{\log_2{\sizeV}} - \ceilbrace{\log_2{k}}+2k-1)+k\sizeV^2 \\
	     %	& & \ \ \ + (2k\sizeV - ( (\sizeV-2)(\sizeV-1) - (\sizeV-k)(\sizeV-k+1) ) ) + 1 \\
		&=& k\sizeV^2 +k^2 +3\sizeV +k +\ceilbrace{\log_2{\sizeV}} - \ceilbrace{\log_2{k}}, \mbox{ and } \\
UB(\sizeV,k) %	&=& (2\sizeV-1) + k\sizeV^2 + (k(2\sizeV-1)+1) + 1\\
		&=& k \sizeV^2 + 2(k+1)\sizeV -k +1.
\end{eqnarray*}
%
\end{lemma}
%
\begin{proof}
 % Since $G=(V,E)$ is $k$-colorable,
 % we have a $k$-partition $V_1, V_2, \ldots, V_k$ of $V$,
 % where each $V_i$ consists of the vertices of the same color.
 % Note that each $V_i$ is an independent set.
 (i) Tracing $M_G$ according to $S_G$, we remove states and transition which is not visited from $M_G$.
 Then we convert from $M_G$ to $\tilde{M}_G$ by Property~\ref{pro:head} - \ref{pro:tail}.
 So it is obviously done by log-space.
 
 (ii) In $\tilde{M}_G$, the numbers of states in head and adjacency coding are denoted by $\Head(\tilde{M}_G)$ and $\AC(\tilde{M}_G)$ respectively.
 The number of states in binary coding and tail is denoted by $\Edge(\tilde{M}_G)$.
 They are bounded as follows.
 \begin{eqnarray*}
  \ceilbrace{\log_2{\sizeV}} - \ceilbrace{\log_2{k}} + 2k \leq & \Edge(\tilde{M}_G) & \leq 2\sizeV \\
  & \Head(\tilde{M}_G) & = k\sizeV^2 \\
  2k\sizeV - \left((\sizeV-2)(\sizeV-1) - (\sizeV-k)(\sizeV-k+1)\right) \leq & AC(\tilde{M}_G) & \leq k(2\sizeV-1) + 1
 \end{eqnarray*}
 Since $|\tilde{M}_G| = \Edge(\tilde{M}_G) + \Head(\tilde{M}_G) + AC(\tilde{M}_G)$, 
 we get the inequality. \qed
 %
 \begin{figure}[tb]
  \begin{center}
   % \includegraphics[width=110mm]{figure/mg.eps}
   % \includegraphics[width=110mm]{figure/mg_merged.eps}
   % \includegraphics[width=70mm]{figure/exposition.eps}
   \includegraphics[width=120mm]{figure/mg.png}
   \includegraphics[width=120mm]{figure/mg_merged.png}
   \includegraphics[width=80mm]{figure/exposition.png}
  \end{center}
  \caption{DFAs that are consistent with $S_G$ in Fig.~\ref{fig:exampleSg2};
  the obvious one (top) and a smaller DFA $\tilde{M}_G$ (bottom),
  which implies that the original graph $G$ is $3$-colorable.}
  % Left is obviously consist $M_G$,
  % and right is a partitioning DFA $\tilde{M}_G$ which is $3$-colorable.}
  \label{fig:minimizeDFA}
 \end{figure}
\end{proof}
% 
% 
\begin{theorem}\label{theorem:intractability}
%	$\MinConDFA$ is $\classNP$-hard, even if $\domain(S) \subseteq
 %	\Prefix(s)$ for some single string $s \in \Sigma^*$.
	$\MinConDFAprefSet$ is $\classNP$-hard.
\end{theorem}
\begin{proof}
	In Lemma~\ref{lemma:consistent DFA UB and LB, version 2}, we
 showed a reduction from \MinGC\ to \\ \MinConDFAprefSet.
 %
	We can translate both from a graph $G$ to the sample $S_G$, and from a DFA $M_G$ to the coloring $\Phi$ in $O(\sizeS)=O(\sizeV^3)$ time.
	The former translation is obvious from the construction of $S_G$,
	 and the latter is done by checking the necks of $M_G$;
	if the neck of $i$'s stream is identical to the one of $j$'s
 stream, the vertices $i$ and $j$ have the same color.
%
	Thus, there is an approximation preserving reduction from \MinGC\ to $\MinConDFAprefSet$.
%	Since \MinGC\ is $\classNP$-hard~\cite{Garey-Johnson-the-black-book-1979},
%	$\MinConDFAprefSet$ is also $\classNP$-hard. 
\qed
\end{proof}


\subsection{Non-approximability result.}


%\subsection{Minimization problems and polynomial-time approximation algorithms}

A {\em polynomial-time approximation algorithm $A$} for a minimization problem $\Pi$ is an algorithm that, for any problem instance $x$ of $\Pi$, produces a feasible solution $s$ in polynomial time with respect to $|x|$. 
For a rational number $r \geq 1$, $A$ {\em guarantees approximation ratio $r$} (or {\em achieves ratio $r$}) if, for any instance, $A$ outputs a solution $s$ satisfying $\frac{|s|}{|s^*|} \leq r$, where $s^*$ is an optimal solution for instance $x$. 

%\snote{
%A minimization problem $\Pi$ is a combinatorial optimization problem (See {\it e.g.} \cite{Ausiello-et-al-book-1999}) 
%that is defined by 
%%Ausiello, G., Crescenzi, P., Gambosi, G., Kann, V., Marchetti-Spaccamela, A., Protasi, M., Complexity and Approximation, Springer, 1999.
%%
%\begin{enumerate}
%\item
%The set $I_\Pi$ of instance,
%\item
%The set $S_\Pi(x)$ of ({\em feasible}) {\em solutions} for any specified instance $x \in I_\Pi$, 
%\item
%The {\em measure} ({\em cost function}) $m_\Pi: I_\Pi \times S_\Pi(x) \to \mathbb{Z}^+$, 
%\end{enumerate}
%and optimization directive `minimize.' 
%We require all the functions specified above is computable in polynomial time with respect to the length of its inputs. 
%Therefore, for a pair of an instance of the minimization problem and a positive integer threshold, 
% the question that asks whether a solution whose measure is no more than the threshold exists
%is a language (decision problem) in class $\classNP$. 
%A solution of $x$ whose measure is minimum in $S_\Pi(x)$ is an {\em optimum solution}. 
%With the notation $m^*_\Pi(x)$ we refer to the measure of an optimum solution of $x$. 
%}


%\end{document}


\begin{theorem}\label{theorem:nonapproximability}
%	$\MinConDFA$ cannot be approximated within ratio $n^{\frac{1}{21}-c}$ for any $c >0$ unless $\classP = \classNP$,
% 	even if $\domain(S) \subseteq \Prefix(s)$ for some single string $s \in \Sigma^*$.
 $\MinConDFAprefSet$ cannot be approximated within ratio $n^{\frac{1}{21}-c}$ for any $c >0$, unless $\classP = \classNP$.
\end{theorem}

\begin{proof}
	If an algorithm for $\MinConDFA$ achieves ratio $r$, it outputs, for any labeled string $S_G$ transformed from $G$, a DFA $M_G$ such that $\frac{|M_G|}{|M^*_G|} \leq r$ with respect to a DFA $M^*_G$ with the minimum number of states.
	By Lemma~\ref{lemma:consistent DFA UB and LB, version 2}, we can translate DFA $M_G$ and $M^*_G$ to a $k$-coloring and $\kappa^*$-coloring, respectively, in polynomial-time. 
	For any $k > \kappa^* \geq 1$ and a sufficiently large $\sizeV$, the ratios satisfy the inequality 
	\[
	\frac{1}{2}\cdot\frac{k}{\kappa^*} <
	\frac{k \, ( {\sizeV}^2 +1)}{ \kappa^* (\sizeV+2)^2 } <
% \textcolor{red}{\frac{\leq k\sizeV^2 + k + (k^2+ 3 \sizeV + \lceil \log_2 \sizeV \rceil - \lceil\log_2 k\rceil) }{\geq \kappa^* \sizeV^2 + 2 \kappa^*(1 +\frac{1}{\kappa^*}) \sizeV -\kappa^* +1} =}
	\frac{LB(\sizeV, k)}{UB(\sizeV, \kappa^*)} \leq
	\frac{|M_G|}{|M^*_G|} \leq r \,.
	\]

	It is known in \cite{Bellare-et-al-siamjc98} (Theorem~\ref{thm:GC is inapproximatability}) that no polynomial-time algorithm can guarantee the ratio $\frac{k}{\kappa^*} < \sizeV^{\frac{1}{7} -c}$ for any $0 < c \leq \frac{1}{7}$, unless $\classP=\classNP$. 
Since $\sizeS$ is $O(\sizeV^3)$, there exists $c^\prime > \frac{1}{3}c$ such that  
%\snote{$\frac{1}{3}c + \delta = c'$ such that} 
the inequality 
	\begin{eqnarray*}
	 \sizeS^{\frac{1}{21}-c'} 
%	<\textcolor{red}{\frac{1}{2} \cdot a^{-\frac{1}{21}} \cdot m^{\delta} \cdot m^{\frac{1}{21}-(\frac{1}{3}c +\delta)} <
%	 \frac{1}{2} \cdot \left(\frac{1}{a} \cdot\sizeS \right)^{\frac{1}{21} - \frac{1}{3}c} 
	 < \frac{1}{2} \cdot \sizeV^{\frac{1}{7}-c} \leq %\,}
 \frac{1}{2}\cdot\frac{k}{\kappa^*} \leq r 
	\end{eqnarray*}
holds. 
%\snote{Because $\frac{1}{2}a^{-\frac{1}{21}} \cdot m^\delta \geq 1$ for sufficiently large $m > n^3$.
%	Thus the theorem holds.}
\qed
\end{proof}

