\documentclass[11pt,sort]{jarticle}
\usepackage{latexsym}
\usepackage{mathrsfs}
%\usepackage{url}
%\usepackage{lscape}
\usepackage{graphics}
\usepackage{theorem}

\newtheorem{caution}{注意}

\input{A4}
\usepackage{theorem}
\renewcommand{\baselinestretch}{1.2}
\setlength{\parskip}{0.25ex}
\renewcommand{\arraystretch}{0.85}

\title{電流と電圧の測定}
\author{下薗 真一}
\date{暫定板 27 Apr. 2015}

\begin{document}

\maketitle

\section{マルチメータ(テスタ)}

交流及び直流の電圧,電流の測定,その他導通の確認や抵抗値の測定を一台でできるようにした装置を,
マルチメータ multimeter(あるいはテスタ,回路計)という.
電気電子回路の作成や電器製品の点検修理などに便利,あるいはなくてはならないものである.
軽量小型,安価で気軽に使えるものから,
精密測定用の有効数字5桁以上でデータ記録機能をそなえたものまで,多種多様である.

ここでは,電気回路の電圧,電流の測定方法の基礎,マルチメータの使用法の基礎を身につけることを目的とする.
回路の組立などで重宝する導通テスタ,抵抗計の使い方,一般的な電圧,電流の測定方法と注意点を学ぶ.

\section{動作原理と注意事項}
多くの場合,アナログテスタは電流計を,デジタルテスタは電圧計を基本としている.
直流,交流それぞれの電圧あるいは電流を,さまざまな測定範囲(レンジ)で測定できるよう,
電流計あるいは電圧計に回路を付加し,測定用プローブの差しかえやセレクタスイッチにより切り替えて使用できるようになっている.
ほとんどの機種で,電池など電源を内蔵して簡易の抵抗計として使用できるようになっており,機種によってはコンデンサ capacitor の静電容量の簡易測定や半導体の動作確認ができるものもある.

\subsubsection*{一般的な注意}
測定レンジを選ぶにあたっては,許容をはるかに超えた電圧や電流を測定しようとすると,機器に損傷をあたえることがあるので,注意すること.
測定する値をできるだけ正確に読み取るためにはできるだけ小さいレンジを選択するが,
測定値が予想できないときは,大きな値のレンジで測定を行ってから,
より小さな値を測定するレンジにきりかえていく.


電流測定モードで電圧計の接続をすると,大電流が流れ,テスタのヒューズを焼き切ったり,機器に損傷をあたえることがあるので,注意すること.
電流測定の場合,測定する回路の電流の流れをさまたげないよう,テスタのプローブ間の抵抗値は非常に小さくなる.
この状態で電位差のあるところにプローブをあてると,大電流が流れることになる.
電流計モードで使用する場合には,特に注意すること.

\subsection{アナログ針式}
アナログ針式テスタの電流計は,コイルを流れる電流によって生じる電磁力で針が振れ,値が読み取れるようになっている.
これに,分流回路や分圧回路を組み合わせて,測定範囲を広げたり,電圧を測定する.
針の指す値は目盛板の目盛(スケール)で読み取る.
電圧または電流のモード,測定範囲などによって異なるスケールを使うため,目盛板には複数のスケールがつけられている.

測定で流れる電流で駆動するので,増幅回路を内蔵するものを除き,基本的に電源を必要としない.
抵抗値は内蔵する電池によって回路をつくり流れる電流から測定するようになっている.

\subsubsection*{注意}
レンジごとにことなるスケールを使用するので,レンジにあわせ適切な目盛を選ぶ.
最小目盛が表す値をあらかじめ確認しておく.
また,針と目盛板の間にはわずかに距離があるので,読み取りの際に視差 parallax が生じないように注意する.
機種によっては,目盛板に鏡がついており,視差をなくせるようになっている.
機械的な機構をもつので,取扱説明書に従い,測定の際は水平な振動しない場所等におく必要がある.

よみとった値には,テスタ自身の表示誤差も含まれる.
測定範囲に対する割合で表される振れ量の誤差と,スケールに対するずれの誤差があり,これらは取扱説明書に記載されているので,確認する.


\subsection{デジタル式}
デジタルテスタは,原理的には電子回路による増幅とA/D変換を行い,数値化した値を表示するもので,
これに回路をくみあわせ,直流だけでなく交流,また電流,その他さまざまな値が測定できるようになっている.
一般に,測定にごくわずかな電流しかながさなくてよい(入力インピーダンス(信号抵抗)が高い)ため,
測定によって回路に影響をあたえにくく,アナログ針式にくらべ非常に小さな電流や電圧を測定できる.
また,A/D 変換の精度をあげて,非常に測定精度の高いものをつくることができる.

測定範囲を自動切り替えするものもあるが,誤使用をふせぐ目的もふくめ,
アナログ針式と同様にレンジをセレクタで切り替えたり,プローブを差しかえて使用する.

多様な電子回路が付加されており,
抵抗値のほか,コンデンサの容量やコイルのインダクタンスなど,さまざまな測定ができるものもある.
増幅回路や表示部を駆動する必要があるため,電源は必ず必要である.

\subsection{測定誤差}

どちらの方式でも,レンジごとの誤差と,針またはデジタル数値の表示誤差があり,
測定値の誤差はその和となる.
これらは,機器の取扱説明書などで確認する必要がある.

\newpage
\section{実験}

乾電池を電源とし,さまざまな抵抗値の抵抗器を接続して,抵抗器に印加される電圧と回路に流れる電流を測定する.
乾電池のモデルを,理想電源と一定値の内部抵抗からなるものとし,
測定結果から内部抵抗の値を求める.

\subsection{内部抵抗の推定}\label{a}

乾電池を,一定の電圧 $V_0$ を出力する理想電池と,一定の抵抗値 $r$ の内部抵抗からなると考える.

\begin{enumerate}
\item
抵抗値が数 $\Omega$ から 数百 $\Omega$ の抵抗器を 8 〜 10 種類選び,
これらの抵抗器を乾電池の負荷として接続し,
抵抗器 $R_i$ を接続したときに回路に流れる電流 $I_i$ と抵抗器への印加電圧 $V_i$ を,各抵抗器について測定せよ.
\item
電流計と電圧計のインピーダンスをそれぞれ $0 \Omega$, $\infty \Omega$ と考え,測定結果 $V_i, I_i$ の組について,関係
\[
V_i = V_0 - r I_i
\]
をグラフにプロットし,その傾きと切片から電池の内部抵抗 $r$ と理想電池の出力電圧 $V_0$ を求めよ.
\end{enumerate}


\subsection{半導体素子の性質}\label{b}

LED を題材に,オームの法則の成り立たない半導体素子のふるまいについてしらべる.

\bibliographystyle{plain}
\bibliography{references}


\newpage
\appendix
\noindent
{\LARGE\bf 付録 Appendix}

\section{デジタルマルチメータの操作}

\subsection{携帯型デジタルマルチメータ}
(例)三和電気計器(株)CD770/CD771

\subsection{携帯型アナログマルチメータ}
(例)三和電気計器(株)

\subsection{据置型多桁デジタルマルチメータ}
(例)

\end{document}
