\documentclass[11pt,sort]{jarticle}
\usepackage{latexsym}
\usepackage{mathrsfs}
%\usepackage{url}
%\usepackage{lscape}
\usepackage{graphics}
\usepackage{theorem}

\newtheorem{caution}{注意}

\input{A4}
\usepackage{theorem}
\renewcommand{\baselinestretch}{1.2}
\setlength{\parskip}{0.25ex}
\renewcommand{\arraystretch}{0.85}

\title{電流と電圧の測定}
\author{下薗 真一}
\date{暫定板 27 Apr. 2015}

\begin{document}

\maketitle

\section{マルチメータ(テスタ)}

交流及び直流の電圧,電流の測定,その他導通の確認や抵抗値の測定を一台でできるようにした装置を,
マルチメータ multimeter(あるいはテスタ,回路計)という.
電気電子回路の作成,電器製品の不具合をみつけるとき,などに便利,あるいはなくてはならないものである.
軽量小型,安価で気軽に使えるものから,
有効数字が5桁以上でデータ記録機能をそなえた研究や精密測定用のものまで,多種多様である.

ここでは,電気回路の電圧,電流の測定方法の基礎,マルチメータの使用法の基礎をみにつけることを目的とする.

\section{測定の原理}
\subsection{アナログ針式}
一般に,アナログテスタは電流計を,デジタルテスタは電圧計を基本としている.
直流,交流それぞれの電圧あるいは電流のさまざまな測定範囲(レンジ)に対応するため,
回路や電源を内蔵し,測定用プローブの差しかえやセレクタスイッチによる回路の切り替え
を行い,他用途に使用できるようになっている.

アナログ針式テスタの電流計は,コイル流れる電流によって生じる力で針が振れるようになっている.
これに,分流回路や分圧回路を組み合わせ,測定範囲を広げたり,電圧を測定できるようにしてある.
測定で流れる電流で駆動するので,増幅回路を内蔵するものを除き,基本的に電源を必要としない.
抵抗値は内蔵する電池によって回路をつくり流れる電流から測定するようになっている.

アナログ針式は一つで目盛板にはレンジごとにことなる目盛がつけられている.
レンジにあわせ適切な目盛で読み取る必要がある.
また,針と目盛板の間にはわずかに距離があるので,読み取りの際に視差が生じないように注意する.
機械的な機構をもつので,測定の際は水平な振動しない場所におく必要がある.

\subsection{デジタル式}
デジタルテスタは,原理的には電子回路による増幅とA/D変換を行い,数値化した値を表示するもので,
これに回路をくみあわせ,直流だけでなく交流,また電流,その他さまざまな値が測定できるようになっている.
一般に,測定にごくわずかな電流しかながさなくてよい(入力インピーダンス(信号抵抗)が高い)ため,
測定によって回路に影響をあたえにくく,アナログ針式にくらべ非常に小さな電流や電圧を測定できる.
また,A/D 変換の精度をあげて,非常に測定精度の高いものをつくることができる.

測定範囲を自動切り替えするものもあるが,誤使用をふせぐ目的もふくめ,
アナログ針式と同様にレンジをセレクタで切り替えたり,プローブを差しかえて使用する.

多様な電子回路が付加されており,
抵抗値のほか,コンデンサの容量やコイルのインダクタンスなど,さまざまな測定ができるものもある.
増幅回路や表示部を駆動する必要があるため,電源は必ず必要である.

\subsection{測定誤差}

どちらの方式でも,レンジごとの誤差と,針またはデジタル数値の表示誤差があり,
測定値の誤差はその和となる.
これらは,機器の取扱説明書などで確認する必要がある.


\section{実験}

\subsection{直流電流と電圧の測定}\label{a}

電源と抵抗器による簡単な回路を作り,
回路を作るうえでのテスターの活用方法と,
電流および電圧の測定の基本を学ぶ.

\subsection{半導体素子の性質}\label{b}

LED を題材に,オームの法則の成り立たない半導体素子のふるまいについてしらべる.

\bibliographystyle{plain}
\bibliography{references}


\newpage
\appendix
\noindent
{\LARGE\bf 付録 Appendix}

\section{デジタルマルチメータの操作}

\subsection{携帯型デジタルマルチメータ}
(例)三和電気計器(株)CD770/CD771

\subsection{携帯型アナログマルチメータ}
(例)三和電気計器(株)

\subsection{据置型多桁デジタルマルチメータ}
(例)

\end{document}
