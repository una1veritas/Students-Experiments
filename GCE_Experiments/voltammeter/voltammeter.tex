\documentclass[11pt,sort]{jarticle}
\usepackage{latexsym}
\usepackage{mathrsfs}
%\usepackage{url}
%\usepackage{lscape}
\usepackage{graphics}
\usepackage{theorem}

\newtheorem{caution}{注意}

%% for apple LaserWriter Series %%
%% 
\setlength{\topmargin}{-0.5in}
\setlength{\textwidth}{5.6in}
\setlength{\textheight}{8.8in}
\setlength{\oddsidemargin}{0.35in}
\setlength{\evensidemargin}{0in}

\usepackage{theorem}
\renewcommand{\baselinestretch}{1.2}
\setlength{\parskip}{0.25ex}
\renewcommand{\arraystretch}{0.85}

\title{電流と電圧の測定}
\author{下薗 真一}
\date{暫定板 27 Apr. 2015}

\begin{document}

\maketitle

\section{マルチメータ(テスタ)}

交流及び直流の電圧,電流の測定,その他導通の確認や抵抗値の測定を一台でできるようにした装置を,
マルチメータ multimeter(あるいはテスタ,回路計)という.
電気電子回路の作成,電器製品の不具合をみつけるとき,などに便利,あるいはなくてはならないものである.
軽量小型,安価で気軽に使えるものから,
有効数字が5桁以上でデータ記録機能をそなえた研究や精密測定用のものまで,多種多様である.

ここでは,電気回路の電圧,電流の測定方法の基礎,マルチメータの使用法の基礎をみにつけることを目的とする.

\section{測定の原理}
一般に,アナログテスタは電流計,デジタルテスタは電圧計が基本となっており,
直流,交流それぞれの電圧あるいは電流をさまざまな測定範囲(レンジ)で使えるよう,回路や電源を内蔵し,測定用プローブの差しかえやセレクタスイッチによる切り替えで多くの用途に使用できるよう作られている.

\subsection{アナログ針式}
アナログ針式テスタの電流計は,コイル流れる電流によって生じる力で針が振れるようになっている.
これに分流回路や分圧回路を組み合わせ,測定範囲を広げたり,電圧を測定できるようにしてある.
測定で流れる電流で針を動かすので,増幅回路を内蔵するものを除き,基本的に電源を必要としない.
抵抗値の測定は,内蔵する電池によって回路をつくり流れる電流から測定する.
この電池の起電力は変化するから,測定の際に 0 オーム調整のつまみで調整してから測定するようになっている.

目盛板にはレンジごとに異なる目盛がつけられており,レンジにあわせて適切な目盛を使い読み取る必要がある.
また,針と目盛板の間にはわずかに距離があるので,読み取りの際に視差が生じないように注意する.
機械的な機構をもつので,測定の際は取り扱い説明書に従い,水平な振動しない場所で使用する.

\subsection{デジタル式}
デジタルテスタは,電子回路による増幅とA/D変換を行い,数値化した値を文字表示する.
アナログ針式と同様に,回路をくみあわせ,直流だけでなく交流,電流,その他さまざまな値が測定できるようになっている.
一般に,測定のために流さなくてはいけない電流はごくわずかである(入力インピーダンス(信号抵抗)が高い)ため,
測定による影響は小さくてすみ,アナログ針式にくらべて非常に小さな電流や電圧を測定できる.
また,A/D 変換の精度をあげて測定精度の非常に高いものがつくられている.

電圧の測定などではレンジを自動切り替えするものもあるが,測定モードはアナログ針式と同様にセレクタで切り替えたり,プローブを差しかえて使用する.

多様な電子回路が付加され,
抵抗値のほか,コンデンサの容量やコイルのインダクタンスなど,さまざまな測定ができるものもある.
増幅回路や表示部を駆動する必要があるため,電源は必ず必要である.

\subsection{測定誤差}

どちらの方式でも,レンジごとの誤差と,針またはデジタル数値の表示誤差があり,
測定値の誤差はその0和となる.
これらは,機器の取扱説明書などで確認する必要がある.

\newpage
\section{実験}

乾電池を電源とし,さまざまな抵抗値の抵抗器を接続して,抵抗器に印加される電圧と回路に流れる電流を測定する.
乾電池のモデルを,理想電源と一定値の内部抵抗からなるものとし,
測定結果から内部抵抗の値を求める.

\subsection{内部抵抗の推定}\label{a}

乾電池を,一定の電圧 $V_0$ を出力する理想電池と,一定の抵抗値 $r$ の内部抵抗からなると考える.

\begin{enumerate}
\item
抵抗値が数 $\Omega$ から 数百 $\Omega$ の抵抗器を 8 〜 10 種類選び,
これらの抵抗器を乾電池の負荷として接続し,
抵抗器 $R_i$ を接続したときに回路に流れる電流 $I_i$ と抵抗器への印加電圧 $V_i$ を,各抵抗器について測定せよ.
\item
電流計と電圧計のインピーダンスをそれぞれ $0 \Omega$, $\infty \Omega$ と考え,測定結果 $V_i, I_i$ の組について,関係
\[
V_i = V_0 - r I_i
\]
をグラフにプロットし,その傾きと切片から電池の内部抵抗 $r$ と理想電池の出力電圧 $V_0$ を求めよ.
\end{enumerate}

<<<<<<< HEAD
乾電池を電源とし,抵抗器などの負荷に電流を流す簡単な回路を作り,
電流および電圧の測定の基本を学ぶ.
=======
>>>>>>> master

新しいアルカリマンガン乾電池,十分に充電されたニッケル水素充電池を電源として,以下を行う.
\begin{enumerate}
\item
抵抗値が数オームから数百キロオームの間の抵抗器を 8 種類ほど選び,
それぞれを負荷としたときの電池の起電力(負荷への印可電圧)と回路の電流の関係を測定せよ.
抵抗器の抵抗値は,マルチメータの抵抗値測定機能を使って確認すること.

また,測定結果をグラフに描け.
\item
乾電池は理想的な電源と内部抵抗の直列からなるというモデルに基づき,
上記の関係から,各電池の内部抵抗を見積もれ.

また,電池の内部抵抗は一定であると考えてよいかどうか,考察せよ.
\item
豆電球,おもちゃのモーター,その他一般的な負荷に対してどれくらいの電流が流れるかをしらべ,電池の寿命を推測する場合に適切な負荷の抵抗値を結論せよ.
\end{enumerate}

\subsection{半導体素子の性質}\label{b}

LED を題材に,オームの法則の成り立たない半導体素子のふるまいについてしらべる.

\bibliographystyle{plain}
\bibliography{references}


\newpage
\appendix
\noindent
{\LARGE\bf 付録 Appendix}

\section{デジタルマルチメータの操作}

\subsection{携帯型デジタルマルチメータ}
(例)三和電気計器(株)CD770/CD771

\subsection{携帯型アナログマルチメータ}
(例)三和電気計器(株)

\subsection{据置型多桁デジタルマルチメータ}
(例)

\end{document}
